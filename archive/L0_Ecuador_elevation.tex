% Options for packages loaded elsewhere
\PassOptionsToPackage{unicode}{hyperref}
\PassOptionsToPackage{hyphens}{url}
%
\documentclass[
]{article}
\usepackage{amsmath,amssymb}
\usepackage{lmodern}
\usepackage{iftex}
\ifPDFTeX
  \usepackage[T1]{fontenc}
  \usepackage[utf8]{inputenc}
  \usepackage{textcomp} % provide euro and other symbols
\else % if luatex or xetex
  \usepackage{unicode-math}
  \defaultfontfeatures{Scale=MatchLowercase}
  \defaultfontfeatures[\rmfamily]{Ligatures=TeX,Scale=1}
\fi
% Use upquote if available, for straight quotes in verbatim environments
\IfFileExists{upquote.sty}{\usepackage{upquote}}{}
\IfFileExists{microtype.sty}{% use microtype if available
  \usepackage[]{microtype}
  \UseMicrotypeSet[protrusion]{basicmath} % disable protrusion for tt fonts
}{}
\makeatletter
\@ifundefined{KOMAClassName}{% if non-KOMA class
  \IfFileExists{parskip.sty}{%
    \usepackage{parskip}
  }{% else
    \setlength{\parindent}{0pt}
    \setlength{\parskip}{6pt plus 2pt minus 1pt}}
}{% if KOMA class
  \KOMAoptions{parskip=half}}
\makeatother
\usepackage{xcolor}
\usepackage[margin=1in]{geometry}
\usepackage{color}
\usepackage{fancyvrb}
\newcommand{\VerbBar}{|}
\newcommand{\VERB}{\Verb[commandchars=\\\{\}]}
\DefineVerbatimEnvironment{Highlighting}{Verbatim}{commandchars=\\\{\}}
% Add ',fontsize=\small' for more characters per line
\usepackage{framed}
\definecolor{shadecolor}{RGB}{248,248,248}
\newenvironment{Shaded}{\begin{snugshade}}{\end{snugshade}}
\newcommand{\AlertTok}[1]{\textcolor[rgb]{0.94,0.16,0.16}{#1}}
\newcommand{\AnnotationTok}[1]{\textcolor[rgb]{0.56,0.35,0.01}{\textbf{\textit{#1}}}}
\newcommand{\AttributeTok}[1]{\textcolor[rgb]{0.77,0.63,0.00}{#1}}
\newcommand{\BaseNTok}[1]{\textcolor[rgb]{0.00,0.00,0.81}{#1}}
\newcommand{\BuiltInTok}[1]{#1}
\newcommand{\CharTok}[1]{\textcolor[rgb]{0.31,0.60,0.02}{#1}}
\newcommand{\CommentTok}[1]{\textcolor[rgb]{0.56,0.35,0.01}{\textit{#1}}}
\newcommand{\CommentVarTok}[1]{\textcolor[rgb]{0.56,0.35,0.01}{\textbf{\textit{#1}}}}
\newcommand{\ConstantTok}[1]{\textcolor[rgb]{0.00,0.00,0.00}{#1}}
\newcommand{\ControlFlowTok}[1]{\textcolor[rgb]{0.13,0.29,0.53}{\textbf{#1}}}
\newcommand{\DataTypeTok}[1]{\textcolor[rgb]{0.13,0.29,0.53}{#1}}
\newcommand{\DecValTok}[1]{\textcolor[rgb]{0.00,0.00,0.81}{#1}}
\newcommand{\DocumentationTok}[1]{\textcolor[rgb]{0.56,0.35,0.01}{\textbf{\textit{#1}}}}
\newcommand{\ErrorTok}[1]{\textcolor[rgb]{0.64,0.00,0.00}{\textbf{#1}}}
\newcommand{\ExtensionTok}[1]{#1}
\newcommand{\FloatTok}[1]{\textcolor[rgb]{0.00,0.00,0.81}{#1}}
\newcommand{\FunctionTok}[1]{\textcolor[rgb]{0.00,0.00,0.00}{#1}}
\newcommand{\ImportTok}[1]{#1}
\newcommand{\InformationTok}[1]{\textcolor[rgb]{0.56,0.35,0.01}{\textbf{\textit{#1}}}}
\newcommand{\KeywordTok}[1]{\textcolor[rgb]{0.13,0.29,0.53}{\textbf{#1}}}
\newcommand{\NormalTok}[1]{#1}
\newcommand{\OperatorTok}[1]{\textcolor[rgb]{0.81,0.36,0.00}{\textbf{#1}}}
\newcommand{\OtherTok}[1]{\textcolor[rgb]{0.56,0.35,0.01}{#1}}
\newcommand{\PreprocessorTok}[1]{\textcolor[rgb]{0.56,0.35,0.01}{\textit{#1}}}
\newcommand{\RegionMarkerTok}[1]{#1}
\newcommand{\SpecialCharTok}[1]{\textcolor[rgb]{0.00,0.00,0.00}{#1}}
\newcommand{\SpecialStringTok}[1]{\textcolor[rgb]{0.31,0.60,0.02}{#1}}
\newcommand{\StringTok}[1]{\textcolor[rgb]{0.31,0.60,0.02}{#1}}
\newcommand{\VariableTok}[1]{\textcolor[rgb]{0.00,0.00,0.00}{#1}}
\newcommand{\VerbatimStringTok}[1]{\textcolor[rgb]{0.31,0.60,0.02}{#1}}
\newcommand{\WarningTok}[1]{\textcolor[rgb]{0.56,0.35,0.01}{\textbf{\textit{#1}}}}
\usepackage{graphicx}
\makeatletter
\def\maxwidth{\ifdim\Gin@nat@width>\linewidth\linewidth\else\Gin@nat@width\fi}
\def\maxheight{\ifdim\Gin@nat@height>\textheight\textheight\else\Gin@nat@height\fi}
\makeatother
% Scale images if necessary, so that they will not overflow the page
% margins by default, and it is still possible to overwrite the defaults
% using explicit options in \includegraphics[width, height, ...]{}
\setkeys{Gin}{width=\maxwidth,height=\maxheight,keepaspectratio}
% Set default figure placement to htbp
\makeatletter
\def\fps@figure{htbp}
\makeatother
\setlength{\emergencystretch}{3em} % prevent overfull lines
\providecommand{\tightlist}{%
  \setlength{\itemsep}{0pt}\setlength{\parskip}{0pt}}
\setcounter{secnumdepth}{-\maxdimen} % remove section numbering
\ifLuaTeX
  \usepackage{selnolig}  % disable illegal ligatures
\fi
\IfFileExists{bookmark.sty}{\usepackage{bookmark}}{\usepackage{hyperref}}
\IfFileExists{xurl.sty}{\usepackage{xurl}}{} % add URL line breaks if available
\urlstyle{same} % disable monospaced font for URLs
\hypersetup{
  pdftitle={Ecuador Elevation},
  pdfauthor={Hazel J. Anderson},
  hidelinks,
  pdfcreator={LaTeX via pandoc}}

\title{Ecuador Elevation}
\author{Hazel J. Anderson}
\date{2022-10-21}

\begin{document}
\maketitle

\hypertarget{set-file-paths}{%
\section{Set file paths}\label{set-file-paths}}

\begin{Shaded}
\begin{Highlighting}[]
\NormalTok{data\_path}\OtherTok{\textless{}{-}}\FunctionTok{file.path}\NormalTok{(}\StringTok{\textquotesingle{}G:/Shared drives/SpaCE\_Lab\_FRUGIVORIA/data/plants/L0\textquotesingle{}}\NormalTok{)}
\NormalTok{output\_path}\OtherTok{\textless{}{-}} \FunctionTok{file.path}\NormalTok{(}\StringTok{\textquotesingle{}G:/Shared drives/SpaCE\_Lab\_FRUGIVORIA/data/plants/L0\textquotesingle{}}\NormalTok{)}
\end{Highlighting}
\end{Shaded}

\hypertarget{load-required-packages}{%
\section{Load required packages}\label{load-required-packages}}

\begin{Shaded}
\begin{Highlighting}[]
\FunctionTok{library}\NormalTok{(geodata)}
\end{Highlighting}
\end{Shaded}

\begin{verbatim}
## Loading required package: terra
\end{verbatim}

\begin{verbatim}
## terra 1.6.47
\end{verbatim}

\begin{Shaded}
\begin{Highlighting}[]
\FunctionTok{library}\NormalTok{(elevatr)}
\end{Highlighting}
\end{Shaded}

\hypertarget{download-elevation-data-for-ecuador-using-geodata}{%
\section{Download elevation data for Ecuador using
geodata}\label{download-elevation-data-for-ecuador-using-geodata}}

\begin{Shaded}
\begin{Highlighting}[]
\CommentTok{\# get outline of Ecuador}
\NormalTok{ecuador\_outline }\OtherTok{\textless{}{-}} \FunctionTok{gadm}\NormalTok{(}\AttributeTok{country =} \StringTok{"ECU"}\NormalTok{, }\AttributeTok{level=}\DecValTok{0}\NormalTok{, }\AttributeTok{path =}\NormalTok{ data\_path, }\AttributeTok{version=}\StringTok{"latest"}\NormalTok{, }\AttributeTok{resolution=}\DecValTok{1}\NormalTok{)}
\NormalTok{ecuador\_outline}
\end{Highlighting}
\end{Shaded}

\begin{verbatim}
##  class       : SpatVector 
##  geometry    : polygons 
##  dimensions  : 1, 2  (geometries, attributes)
##  extent      : -92.00854, -75.18715, -5.015803, 1.681835  (xmin, xmax, ymin, ymax)
##  coord. ref. : lon/lat WGS 84 (EPSG:4326) 
##  names       : GID_0 COUNTRY
##  type        : <chr>   <chr>
##  values      :   ECU Ecuador
\end{verbatim}

\begin{Shaded}
\begin{Highlighting}[]
\CommentTok{\# get elevation data for Ecuador}
\NormalTok{ecuador\_geodata }\OtherTok{\textless{}{-}} \FunctionTok{elevation\_30s}\NormalTok{(}\AttributeTok{country =} \StringTok{"ECU"}\NormalTok{, }\AttributeTok{path =}\NormalTok{ data\_path)}
\NormalTok{ecuador\_geodata}
\end{Highlighting}
\end{Shaded}

\begin{verbatim}
## class       : SpatRaster 
## dimensions  : 816, 792, 1  (nrow, ncol, nlyr)
## resolution  : 0.008333333, 0.008333333  (x, y)
## extent      : -81.6, -75, -5.2, 1.6  (xmin, xmax, ymin, ymax)
## coord. ref. : lon/lat WGS 84 (EPSG:4326) 
## source      : ECU_elv_msk.tif 
## name        : ECU_elv_msk 
## min value   :          -6 
## max value   :        6169
\end{verbatim}

\begin{Shaded}
\begin{Highlighting}[]
\CommentTok{\# look at elevation data}
\FunctionTok{plot}\NormalTok{(ecuador\_geodata)}
\FunctionTok{plot}\NormalTok{(ecuador\_outline, }\AttributeTok{add =}\NormalTok{ T)}
\end{Highlighting}
\end{Shaded}

\includegraphics{L0_Ecuador_elevation_files/figure-latex/unnamed-chunk-3-1.pdf}

\begin{Shaded}
\begin{Highlighting}[]
\CommentTok{\#save elevation data to file}
\NormalTok{ecuador\_geodata\_elevation\_path }\OtherTok{\textless{}{-}} \FunctionTok{file.path}\NormalTok{(output\_path,}\StringTok{"ecuador\_geodata\_elevation.RData"}\NormalTok{) }
\FunctionTok{save}\NormalTok{(ecuador\_geodata, }\AttributeTok{file =}\NormalTok{ ecuador\_geodata\_elevation\_path)}
\end{Highlighting}
\end{Shaded}

\hypertarget{download-elevation-data-for-ecuador-using-elevatr}{%
\section{Download elevation data for Ecuador using
`elevatr'}\label{download-elevation-data-for-ecuador-using-elevatr}}

\begin{Shaded}
\begin{Highlighting}[]
\CommentTok{\# convert ecuador\_outline to spatial dataframe}
\NormalTok{ecuador\_outline.df }\OtherTok{\textless{}{-}}\NormalTok{ sf}\SpecialCharTok{::}\FunctionTok{st\_as\_sf}\NormalTok{(ecuador\_outline)}

\CommentTok{\# get elevation data for Ecuador}
\NormalTok{ecuador\_elevatr }\OtherTok{\textless{}{-}} \FunctionTok{get\_elev\_raster}\NormalTok{(ecuador\_outline.df, }\AttributeTok{z=} \DecValTok{5}\NormalTok{)}
\end{Highlighting}
\end{Shaded}

\begin{verbatim}
## Mosaicing & Projecting
\end{verbatim}

\begin{verbatim}
## Note: Elevation units are in meters.
\end{verbatim}

\begin{Shaded}
\begin{Highlighting}[]
\CommentTok{\# plot elevation}
\FunctionTok{plot}\NormalTok{(ecuador\_elevatr)}
\FunctionTok{plot}\NormalTok{(ecuador\_outline, }\AttributeTok{add =}\NormalTok{ T)}
\end{Highlighting}
\end{Shaded}

\includegraphics{L0_Ecuador_elevation_files/figure-latex/unnamed-chunk-4-1.pdf}

\begin{Shaded}
\begin{Highlighting}[]
\CommentTok{\#save elevation data to file}
\NormalTok{ecuador\_elevatr\_elevation\_path }\OtherTok{\textless{}{-}} \FunctionTok{file.path}\NormalTok{(output\_path,}\StringTok{"ecuador\_elevatr\_elevation.RData"}\NormalTok{) }
\FunctionTok{save}\NormalTok{(ecuador\_elevatr, }\AttributeTok{file =}\NormalTok{ ecuador\_elevatr\_elevation\_path)}
\end{Highlighting}
\end{Shaded}


\end{document}
